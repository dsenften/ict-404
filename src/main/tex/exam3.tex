\documentclass[12pt,a4paper]{exam}
\usepackage{natbib}

\printanswers % If you want to print answers
%\noprintanswers % If you don't want to print answers

\renewcommand{\solutiontitle}{\noindent\textbf{Lösung:}\enspace}

\RequirePackage{talent-factory}
\RequirePackage[top=3cm,bottom=2cm,left=2cm,right=1.5cm,headsep=10pt,a4paper]{geometry} % Page margins

\newcommand{\class}{Modul 404}
\newcommand{\examnum}{Test 3}
\newcommand{\examdate}{19.03.2019}
\newcommand{\timelimit}{45 Minuten}

\pagestyle{head}
\firstpageheader{}{}{}
\runningheader{\class}{\examnum\ - Seite \thepage\ / \numpages}{\examdate}
\runningheadrule

\begin{document}

    \setmainfont{Verdana}
    \parindent0em\parskip1em

    % Deutsche Schreibweise diverser Labels
    \pointpoints{Punkt}{Punkte}


    \noindent
    \begin{tabular*}{\textwidth}{l @{\extracolsep{\fill}} r @{\extracolsep{6pt}} l}
        \textbf{\class}           & \textbf{Name:} & \makebox[2in]{\hrulefill} \\
        \textbf{\examnum}         & & \\
        \textbf{\examdate}        & & \\
        \textbf{Zeit: \timelimit} & Dozent: & \makebox[2in]{Daniel Senften}
    \end{tabular*}\\
    \rule[1ex]{\textwidth}{1pt}

    Dieser Test besteht aus \numpages\ Seiten (inkl. Titelseite) und
    \numquestions\ Fragen. Wenn alle Fragen richtig beantwortet werden,
    dann sind maximal \numpoints\ möglich.


    \begin{center}
        \textbf{Punktetabelle}\footnote{Diese Tabelle wird durch den Dozenten
        nach der Prüfung ausgefüllt.} \\ \vskip1em
        \addpoints

        \hqword{Aufgabe}
        \hpword{Punkte}
        \htword{\textbf{Total}}
        \hsword{Erreicht}
        \gradetable[h][questions]
    \end{center}

    \noindent
    \rule[1ex]{\textwidth}{1pt}

    % ===========================================================================

    \begin{questions}

        \question[10] Gegeben sind beiden Klassen \textbf{Employee} und
        \textbf{Engineer}. Zeichne das vollständige Klassendiagramm (UML).

        \inputminted[autogobble,frame=single,firstline=10]{java}
        {../java/academy/inheritance/Employee.java}

        \inputminted[autogobble,frame=single,firstline=10]{java}
        {../java/academy/inheritance/Engineer.java}

        \ifprintanswers
        \begin{solution}
            \begin{center}
                \begin{tikzpicture}

                    \umlclass[width=8cm]{Employee}{%
                    numberOfEmployees : int \\
                    empployeeID : int \\
                    skills : List<Competency>}{%
                    addSkills() : void \\
                    getSkills() : List<Competency>}

                    \umlclass[y=-5cm,width=8cm]{Engineer}{%
                    projects : List<String>}
                    {getProjects() : List<String>}

                    \umlVHVinherit{Engineer}{Employee}

                \end{tikzpicture}
            \end{center}
        \end{solution}
        \else\makeemptybox{\fill}
        \fi

        \addpoints

        % -----------------------------------------------
        \newpage
        \question[10] Denken Sie an unser Projekt \emph{Simulation}, in welchem
        wir auch die Klasse \textbf{Position} kennen gelernt haben.

        Wie testen Sie die folgende Funktion auf ihre Richtigkeit?

        % setValue()
        \inputminted[autogobble,firstline=28,lastline=36]
        {java}{../java/simulation/Position.java}

        \ifprintanswers
        \begin{solution}
            \inputminted[autogobble,firstline=14,lastline=22]
            {java}{../../test/java/simulation/PositionTest.java}
            \inputminted[autogobble,firstline=24,lastline=28]
            {java}{../../test/java/simulation/PositionTest.java}
        \end{solution}
        \else\makeemptybox{5cm}
        \fi

        \addpoints

        % -----------------------------------------------
        \question[4] Wir haben das Projekt \emph{Zeit} mit der Darstellung einer
        Zeitangabe kennen gelernt. Hierbei kam auch die folgende Funktion vor:

        \inputminted[autogobble,firstline=70,lastline=74]
        {java}{../java/time/NumberDisplay.java}

        Was würde passieren, wenn wir den Oparator \texttt{<} in der Methode
        \mintinline{java}{setValue()} durch \texttt{<=} ersetzen würden, und zwar auf folgende
        Weise:

        \mintinline{java}{if((value >= 0) && (value <= limit))}

        \ifprintanswers
        \begin{solution}
            Der Wert könnte somit identisch mit dem Wert für \textbf{limit} sein.
            Es wäre somit die Zeitangabe \fbox{24:60} möglich, welche aber nicht
            existiert.
        \end{solution}
        \else\makeemptybox{\fill}\newpage
        \fi

        \addpoints

        % -----------------------------------------------
        \checkboxchar{$\Box$} % changing checkbox style locally
        \question[6] Die boolsche Algebra ist ein sehr wichtiger Bestandteil in der
        Softwareentwicklung. Sie kommt in vielen Ausdrücken wie \textbf{for}-Schleifen,
        \textbf{while}-Schleifen$\ldots$ vor.

        Welche der folgenden Ausdrücke liefern den Wert \texttt{true}?

        \addpoints
        \begin{checkboxes}
            \CorrectChoice \mintinline{java}{(d || (d && f) == d)}
            \CorrectChoice \mintinline{java}{(d && (d || f) == d)}
            \CorrectChoice \mintinline{java}{(x || (y && !y) == x)}
            \CorrectChoice \mintinline{java}{(x && (y || !y) == x)}
            \CorrectChoice \mintinline{java}{s || !s}
            \choice \mintinline{java}{s && !s}
        \end{checkboxes}

        % -----------------------------------------------
        \question[10] Beim unserem Projekt \emph{Time} wird die Zeit im Dezimalformat
        dargestellt, wie in \fbox{14:45}. Ändern Sie diese Anzeige in ein binäres
        Format, so dass diese Zeit als \fbox{01110:101101} dargestellt wird.

        Aktuelle Funktion:

        % updateDisplay()
        \inputminted[autogobble,firstline=75,lastline=78]
        {java}{../java/time/Display.java}
        \inputminted[autogobble,firstline=40,lastline=46]
        {java}{../java/time/NumberDisplay.java}


        \addpoints
        \ifprintanswers
        \begin{solution}
            Dies ist nur eine der Möglichkeiten. Sie sieht etwas komplexer
            aus, da mit dieser Funktion gleichzeitig die maximale
            Anzahl der führenden '0' berücksichtigt wird.
            \inputminted[autogobble,firstline=92,lastline=95]
            {java}{../java/time/Display.java}
            \inputminted[autogobble,firstline=48,lastline=53]
            {java}{../java/time/NumberDisplay.java}
        \end{solution}
        \else\makeemptybox{\fill}\newpage
        \fi


        % -----------------------------------------------
        \question[10]
        Mit der folgender Klasse aus dem Package \textbf{java.util} sollen
        Objekte vom Typ \textbf{Engineer} verwaltet werden. Erstelle eine
        entsprechende Liste und füge anschliessend ein Element mit der
        Funktion \textbf{add()} hinzu.

        \inputminted[autogobble,linenos,firstline=8,lastline=12]
        {java}{../java/collections/ArrayList.java}
        \inputminted[autogobble,linenos,firstline=22]
        {java}{../java/collections/ArrayList.java}

        \ifprintanswers
        \begin{solution}
            \begin{minted}[autogobble]{java}
                ArrayList<Engineer> list = new ArrayList<>();
                list.add(new Engineer("Daniel", "Senften"));
            \end{minted}
        \end{solution}
        \else\makeemptybox{5cm}\newpage
        \fi

        % -----------------------------------------------
        \question
        Objekte und Klassen bauen auf folgenden (nicht abschliessenden) Konzepten auf.
        Was verstehen wir unter den einzelnen Punkten. Gefragt ist eine kurze, stichwortartige
        Umschreibung. Gib für jeden Punkt ein konkretes Beispiel.

        \begin{parts}
            \part[3] Klasse
            \part[3] Objekt
            \part[3] Eigenschaft
            \part[3] Methode
            \part[3] Datentyp
        \end{parts}

        \addpoints

        \ifprintanswers
        \begin{solution}
            Viele Begriffe der objektorientierten Programmierung sind unter
            \href{https://de.wikipedia.org/wiki/Kategorie:Objektorientierte_Programmierung}
            {Wikipedia} sehr gut beschrieben.
            \begin{description}
                \item[Klasse]
                Unter einer Klasse (auch Objekttyp genannt) versteht man
                in der objektorientierten Programmierung ein abstraktes
                Modell bzw. einen Bauplan für eine Reihe von ähnlichen
                Objekten.

                Konkrete Beispiele:
                \begin{minted}[autogobble]{java}
                    public class Person {}
                    public class Employee {}
                \end{minted}

                \item[Objekt] Ein Objekt (auch Instanz genannt) bezeichnet in
                der objektorientierten Programmierung (OOP) ein Exemplar
                eines bestimmten Datentyps oder einer bestimmten Klasse.
                Objekte sind konkrete Ausprägungen („Instanzen“) eines
                Objekttyps und werden während der Laufzeit erzeugt
                (Instanziierung). Sie sind nicht nur zu ihren eigenen
                Klassen, sondern auch zu den entsprechenden Basisklassen
                zuweisungskompatibel.

                In den folgenden Beispielen sind \textbf{person} und \textbf{employee}
                konkrete Objekte (Instanzen) der Klassen \textbf{Person}
                und \textbf{Employee}.
                \begin{minted}[autogobble]{java}
                    Person person = new Person();
                    Employee employee = new Employee();
                \end{minted}

                \item[Eigenschaft] Auch Attribut genannt, gilt im Allgemeinen als
                Merkmal, Kennzeichen, Informationsdetail etc., das einem
                konkreten Objekt zugeordnet ist. Dabei wird unterschieden
                zwischen der Bedeutung (z.B.~``Augenfarbe'') und der konkreten
                Ausprägung (z.B.~``blau'') des Attributs.

                So hat ein Objekt der Klasse \textbf{Person} möglicherweise
                folgende Eigenschaften:

                \begin{minted}[autogobble]{java}
                    private String firstName;
                    private String lastName;
                    private Date birthDate;
                \end{minted}

                \item[Methode] Methoden sind Unterprogramme (in der Form von Funktionen
                oder Prozeduren), die das Verhalten von Objekten beschreiben und
                implementieren. Über die Methoden des Objekts können Objekte
                untereinander in Verbindung treten.

                \begin{minted}[autogobble]{java}
                    public void startEngine() {
                        if (sufficientGasolineAvailable) {
                            // let's start the engine
                        }
                    }
                \end{minted}

                \item[Datentyp] Formal bezeichnet ein Datentyp oder eine Datenart
                in der Informatik die Zusammenfassung von Objektmengen mit
                den darauf definierten Operationen. Dabei werden durch den
                Datentyp des Datensatzes unter Verwendung einer sogenannten
                Signatur ausschliesslich die Namen dieser Objekt- und
                Operationsmengen spezifiziert.

                Hier einige Beispiele:
                \begin{minted}[autogobble]{java}
                    Integer // Datentyp für ganzzahlige Operationen
                    String  // Datentyp für Operationen mit beliebigen Zeichenketten
                \end{minted}

            \end{description}
        \end{solution}
        \else\makeemptybox{\fill}
        \fi


        % -----------------------------------------------
        \newpage
        \question[10]
        Gegeben ist die folgende Klasse:
        \inputminted[autogobble,linenos,firstline=163,lastline=193]
        {java}{../java/simulation/Field.java}

        Was ist das Resultat dieser Funktion, wenn sie mit folgendem
        Ausdruck aufgerufen wird?
        \begin{minted}[autogobble]{java}
            List<Position> liste = adjacentLocations(new Position(0, 12));
        \end{minted}

        \ifprintanswers
        \begin{solution}
            Die Resultat '\textbf{liste}' enthält 5 \textbf{Position}
            Objekte mit den Werten $[0,11]$, $[0,13]$, $[1,11]$, $[1,12]$ und
            $[1,13]$, wobei die Reihenfolge dieser Objekte zufällig angeordnet
            sein wird.
        \end{solution}
        \else\makeemptybox{\fill}
        \fi

    \end{questions}

\end{document}
