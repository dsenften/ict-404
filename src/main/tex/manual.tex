\mode*

\setmonofont{Verdana}
\newmintedfile{java}{frame=single,linenos,samepage}
\newmintedfile{antlr}{linenos}

\section{Java-Technologie kurz erklärt}
\label{sec:java-technologie}

Java ist eine Programmiersprache und Computerplattform, die erstmals 1995 von 
Sun Microsystems veröffentlicht wurde. Es gibt viele Anwendungen und
Websites, die nicht funktionieren, es sei denn, Sie haben Java installiert, 
und jeden Tag werden weitere erstellt. Java ist schnell, sicher und
zuverlässig. Von Laptops bis hin zu Rechenzentren, Spielkonsolen bis hin zu 
wissenschaftlichen Supercomputern, Handys bis hin zum Internet.

Am 20. April 2009 kündigte Oracle die Übernahme von Sun Microsystems für 
7,4 Milliarden US-Dollar an, welche in den Folgemonaten durch verschiedene
Behörden überprüft und anschliessend genehmigt wurde.

Java als Entwicklungs- und Laufzeitumgebung kann in verschiedenen Varianten
von den folgenden Seiten heruntergeladen und installiert werden:

\subsection{Installation der Umgebung}
\label{subsec:installation}

\begin{frame}{Installation der Umgebung}
	Wir unterscheiden im Wesentlichen zwischen den beiden Installationstypen
	JRE ({\em Java Runtime Environment}) und JDK ({\em Java 
	Development Kit}).
	
    \begin{itemize}
		\item \href{https://java.com}{java.com}   
		\item \href{https://openjdk.java.net}{openjdk.java.net}
    \end{itemize}
\end{frame}

In der Softwareentwicklung konzentrieren wir uns ausschliesslich auf das JDK,
da wir auf die vielen zusätzlichen {\em Tools} in der Entwicklung angewiesen
sind.
	
\begin{frame}{Werkzeuge und Befehle}
	Die JDK-Tools und ihre Befehle ermöglichen es uns, 
	Entwicklungsaufgaben wie das Kompilieren und Ausführen eines Programms, 
	das Paketieren von Quelldateien und vieles mehr zu erledigen.
	
	\mode<presentation>{Hier nur eine kleine Auswahl:}
	\begin{itemize}
		\item\href{https://docs.oracle.com/en/java/javase/11/tools/javac.html}
		{javac}---Kompilieren von Klassen- und Schnittstellendefinitionen 
		in Bytecode.
		\item\href{https://docs.oracle.com/en/java/javase/11/tools/java.html}
		{java}---Starten/Interpretieren eines Java Programms.
		\item\href{https://docs.oracle.com/en/java/javase/11/tools/jar.html}
		{jar}---Erstellen eines Java Archives.
		\item\href{https://docs.oracle.com/en/java/javase/11/tools/jshell.html}
		{jshell}---Interaktives Auswerten von Deklarationen, Anweisungen und 
		Ausdrücken der Programmiersprache Java. 
		\item<article>\href{https://docs.oracle.com/en/java/javase/11/tools/javap.html}
		{javap}---Befehl, um eine oder mehrere Klassendateien zu 
		disassemblieren.
		\item<article>\href{https://docs.oracle.com/en/java/javase/11/tools/javadoc.html}
		{javadoc}---Erstellen der Java Dokumentation
		\item<article>\href{https://docs.oracle.com/en/java/javase/11/tools/keytool.html}
		{keytool}---Befehl und Optionen zum Verwalten eines kryptografischer Schlüssel.
		\item<article>\href{https://docs.oracle.com/en/java/javase/11/tools/jarsigner.html}
		{jarsigner}---Signieren und verifizieren von Java Archiven
	\end{itemize}
\end{frame}

Diese Liste stellt keinen Anspruch auf Vollständigkeit. Die komplette Dokumentation
zu den einzelnen Umgebungen und deren {\em Tools} kann jederzeit unter
\href{https://docs.oracle.com/en/java/}{docs.oracle.com} eingesehen werden.

Vorsicht ist allerdings geboten, wenn es um den Einsatz und die Lizenzierung von Java geht.
Oracle wird nach Januar 2019 keine weiteren Updates von Java SE 8 zur 
gewerblichen Nutzung auf seinen Download-Sites veröffentlichen. Kunden, die 
weiterhin Zugriff auf kritische Fehlerkorrekturen und Sicherheitslücken sowie 
eine allgemeine Wartung von Java SE 8 bzw. früheren Versionen benötigen, 
können einen langfristigen Support über Oracle Java SE Advanced erhalten. 
Oracle Java SE Advanced Desktop oder Oracle Java SE Suite. Für weitere 
Informationen und Details, wie Sie einen längerfristigen Support für Oracle 
JDK 8 erhalten, finden Sie in der 
\href{http://www.oracle.com/technetwork/java/eol-135779.html}
{Oracle Java SE Support-Roadmap}.

Die offizielle \href{https://www.oracle.com/corporate/pricing/#java-se}
{Preisliste} hilft, sich mit dem Oracle Java SE Abonnement für On Premise, 
Enterprise, Desktop, Server und Cloud Workloads vertraut zu machen.


\begin{frame}{Beliebte Java-Lernressourcen}
	\begin{itemize}
		\item\textbf{Finch Robot}---Dieser kleine Roboter von 
		\href{https://www.birdbraintechnologies.com/finch/}
		{Bird Technology} nutzt verschiedene Sensoren,
		die via Java angesteuert werden können.
		\item\textbf{Oracle Academy}---Die 
		\href{https://academy.oracle.com/en/training-self-study.html}
		{Oracle Academy} bietet verschiedene 
		Online-Kurse und Anleitungen
		\item\textbf{Scratch}---Dies ist eine sehr einfache, am
		\href{https://scratch.mit.edu/}{MIT} entwickelte Programmierumgebung.
		Typischerweise wird diese im Vorschulalter eingesetzt. 
		\item\textbf{BlueJ}---Als sehr professionelle und stark vereinfachte
		(Entwicklungs-) Umgebung wird \href{http://www.bluej.org/}{BlueJ} 
		auch im späteren Alter noch für Schulungszwecke verwendet.
	\end{itemize}
\end{frame}
	
\begin{Exercise}[%
	title={Installation der Software},
	label={ex:installation}]
	
	Bevor wir mit unseren Übungen starten müssen wir sicherstellen, dass Java und
	seine {\em Tools} komplett und korrekt installiert sind.
	
	Bitte installieren Sie das {\em Java Development Kit} (JDK) gemäss 
	\href{https://jdk.java.net/11/}{Anleitung}.
	
	Anschliessend überprüfen wir die korrekte Installation mit Hilfe der Konsole wie folgt:
	
\begin{minted}[frame=single]{shell}
$ java -version
openjdk version "11.0.1" 2018-10-16
OpenJDK Runtime Environment 18.9 (build 11.0.1+13)
OpenJDK 64-Bit Server VM 18.9 (build 11.0.1+13, mixed mode)
\end{minted}
	
\end{Exercise}	

\subsection{Interaktive Entwicklungsumgebung}

Jeder Java-Entwickler benötigt einen Programmiereditor oder eine IDE, der bei den 
grungierigeren Teilen des Schreibens von Java und der Verwendung von Klassenbibliotheken 
und Frameworks helfen kann. Die Entscheidung, welche IDE zum Einsatz kommt, hängt von 
mehreren Faktoren ab, darunter der Art der zu entwickelnden Projekte, dem vom 
Entwicklungsteam verwendeten Prozess sowie dem Niveau und den Fähigkeiten des 
Entwicklers. Weitere Überlegungen sind, ob das Team die Tools und Ihre persönlichen 
Präferenzen standardisiert hat.

\begin{frame}{Entwicklungsumgebungen}
	Die drei am häufigsten für die Java-Entwicklung gewählten IDEs sind
	
    \begin{itemize}
		\item \href{https://www.jetbrains.com/idea}{IntelliJ IDEA},   
		\item \href{https://www.eclipse.org/}{Eclipse} und
		\item \href{https://netbeans.org/}{NetBeans}
    \end{itemize}
\end{frame}

Ein guter Vergleich dieser drei Umgebungen ist auch
\href{https://www.javaworld.com/article/3114167/development-tools/choosing-your-java-ide.html}
{hier} zu finden. Persönlich habe ich Eclipse in den Anfängen der IDE, später 
NetBeans in den Schulungen und zu guter Letzt IntelliJ IDEA verwendet. Nach 
jedem Wechsel spürte ich, dass sich meine Produktivität verbessert hatte.

Ich habe mich für IntelliJ IDEA Ultimate entschieden. Obwohl es nicht kostenlos 
wie Eclipse oder NetBeans ist, glaube ich, dass der Produktivitätsgewinn den 
Preis wert ist. Für unsere Schulung ist es allerdings unerheblich, welches 
Produkt wir einsetzten, auch wenn ich bestimmt mit IntelliJ IDEA die bessere 
Unterstützung bieten kann.


\section{Analyzing a Problem and Designing a Solution}


\section{Entwicklung eines Java Programms}

\subsection{Meine erstes Programm}

Schauen wir uns nun endlich mal ein Java-Programm im Quelltext an, nachdem 
wir jetzt so viel Theoretisches besprochen haben. Das nachfolgende Programm 
gibt ``Hello World'' aus. Wir werden gleich im Detail besprechen, was die 
einzelnen Zeilen bedeuten. Man muss sich dies allerdings nicht gleich merken 
können, viele Details haben erst später eine grössere Bedeutung.

\mode<presentation>
\begin{frame}{Hello World Java}
	Mein erstes Java Programm.
	\vfill
	\inputminted[%
	frame=single,
	firstline=3]{java}{../java/academy/HelloWorld.java}
\end{frame}

\mode*
\begin{listing}[h]
\javafile{../java/academy/HelloWorld.java}
\caption{Hello World Java}
\label{lst:hello-world}
\end{listing}

Der Quelltext besteht aus sieben Zeilen. Das erste was auffällt ist, 
dass einige Zeilen eingerückt sind. Diese Einrückung dient ausschliesslich 
der Lesbarkeit und ist für den Java Compiler nicht von Bedeutung. 

Was bedeuten die einzelnen Zeilen?

\begin{enumerate}
	\item Java Deteien werden strukturiert in verschiedenen Verzeichnissen
	abgelegt, wobei ein Verzeichnis gleichbedeutend mit einem
	\mintinline{java}{package} ist. In unserem Fall ist die Datei im Verzeichnis 
	`src/main/academy' abgelegt.
	
	\item Leere Zeilen und allgemen Leerzeichen (sog. {\em White Spaces}) werden
	durch den Compiler ignoriert.

	\item Hier wird die Klasse namens \mintinline{java}{HelloWorld} begonnen. 
	Die Klasse ist öffentlich. Dies bedeutet, dass andere Programmteile diese 
	Klasse mit benutzen dürfen. Wenn die im Quelltext beschriebene öffentliche 
	Klasse HelloWorld heisst, dann hat die Datei zwingend den Namen 
	`{\tt HelloWorld.java}', ansonsten meldet der Compiler einen Fehler. 
	
	\item In dieser Zeile wird eine sogenannte Methode mit dem Namen 
	\mintinline{java}{main} angegeben. Methoden sind Funktionseinheiten die 
	man wieder verwenden kann. Auch diese Methode hier ist öffentlich, kann 
	also aus anderen Programmteilen heraus verwendet werden. 
	
	\item An dieser Stelle wird die Zeichenkette "Hello, World!" auf der Konsole 
	ausgegeben. Hierzu verwenden wir die Methode `println()' der Klasse `System'.
	Diese Konstrukte werden wir zu einem späteren Zeitpunkt noch vertieft 
	betrachten.
\end{enumerate}


\subsection{Struktur einer Java Datei}

Java, wie alle anderen Programmiersprachen, regelt seine formale Syntax
sehr genau. Alle Java Dateien folgen einem Aufbau, der teils strikt vorgegeben, 
teils in den Konventionen geregelt ist.

Zur Darstellung der Syntax hat sich die {\em Erweiterte Backus-Naur-Form} 
(\href{https://de.wikipedia.org/wiki/Erweiterte_Backus-Naur-Form}{EBNF})
als Metasprache etabliert.

\mode<presentation>
\begin{frame}{Erweiterte Backus-Naur-Form (EBNF)}
	Eine {\em compilationUnit}, resp. eine Java Datei kann in seiner 
	einfachsten Form wie folgt dargestellt werden:
	\vfill
	\inputminted[%
	firstline=44,lastline=46]{antlr}{Java.g4}
\end{frame}

\mode*
\begin{listing}[H]
\antlrfile[firstline=41,lastline=83]{Java.g4}
\caption{Struktur einer Java Datei}
\label{lst:java-file}
\end{listing}

\section{Deklarieren, initialisieren und verwenden von Variablen}

\section{Erstellen von Objekten}

\begin{listing}[H]
\antlrfile[firstline=90,lastline=107]{Java.g4}
\caption{Deklaration einer Klasse}
\label{lst:java-file}
\end{listing}



\subsection{Das Projekt Lombok}

Mit dem Projekt ``\href{https://projectlombok.org/}{Lombok}'' erreichen wir 
eine Vereinfachung unseres Codes, indem sogenannte Wiederholungen automatisch 
in die Datei `{\tt .class}' generiert werden und die Datei `{\tt .java}' somit 
entlastet und somit lesbarer/wartbarer wird.

Wir versuchen dies in Form einfacher Beispiele zu erläutern.

\textbf{\\@NonNull}

Die Parameter einer Funktion oder eines Konstruktors können bei der Verwendung 
den Wert {\tt null} einnehmen, was möglicherweise einen ungewollten Ablauf 
provozieren kann. Falls wir dies vermeiden wollen, dann bleibt uns nur eine 
entsprechende Fehlerbehandlung. Es ist an dieser Stelle sehr rasch ersichtlich, 
dass dies zwangläufig zu Wiederholungen in der Programmierung führen wird.

\javafile{../java/academy/lombok/nonnull/VanillaJava.java}


\mode<presentation>
\begin{frame}[fragile]{$@$NonNull}
	Bei jeder notwendigen Überprüfung der Argumente einer Funktion oder eines
	Konstruktorst erstellen wir mehr oder weniger den selben Code.
\inputminted[%
	frame=single,
	firstline=5]{java}{../java/academy/lombok/nonnull/VanillaJava.java}
\end{frame}


\mode<presentation>
\begin{frame}[fragile]{$@$NonNull}
	Viel einfacher sieht der Code aus, bei der Verwendung der Annotation 
	\mintinline{java}{@NonNull}.
	\vfill
\inputminted[%
	frame=single,
	firstline=6]{java}{../java/academy/lombok/nonnull/WithLombok.java}
\end{frame}

\mode*

Viel einfacher sieht der Code aus, bei der Verwendung der Annotation 
\mintinline{java}{@NonNull}. Falls wir mit IntelliJ IDEA arbeiten, dann
könnten wir das selbe Resultat mit der vordefinierten Annotation 
\mintinline{java}{@NotNull} von JetBrains erreichen.
	
\begin{listing}[h]
\javafile{../java/academy/lombok/nonnull/WithLombok.java}
\label{lst:lombok-nonnull}
\caption{NullPointer Handling bei der Parameterübergabe}
\end{listing}



\section{Using Operators and Decision Constructs}
\section{Using Loop Constructs}
\section{Developing and Using Methods}
\section{Implementing Encapsulation and Constructors}
\section{Creating and Using Arrays}
\section{Implementing Inheritance}
