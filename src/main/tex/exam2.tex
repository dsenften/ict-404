\documentclass[12pt,a4paper]{exam}
\usepackage{natbib}

%\printanswers % If you want to print answers
%\noprintanswers % If you don't want to print answers

\renewcommand{\solutiontitle}{\noindent\textbf{Lösung:}\enspace}

\RequirePackage{talent-factory}
\RequirePackage[top=3cm,bottom=2cm,left=2cm,right=1.5cm,headsep=10pt,a4paper]{geometry} % Page margins

\newcommand{\class}{Modul 404}
\newcommand{\examnum}{Test 2}
\newcommand{\examdate}{14.03.2019}
\newcommand{\timelimit}{45 Minuten}

\pagestyle{head}
\firstpageheader{}{}{}
\runningheader{\class}{\examnum\ - Seite \thepage\ / \numpages}{\examdate}
\runningheadrule

\begin{document}

    \setmainfont{Verdana}
    \parindent0em\parskip1em

    % Deutsche Schreibweise diverser Labels
    \pointpoints{Punkt}{Punkte}


    \noindent
    \begin{tabular*}{\textwidth}{l @{\extracolsep{\fill}} r @{\extracolsep{6pt}} l}
        \textbf{\class}           & \textbf{Name:} & \makebox[2in]{\hrulefill} \\
        \textbf{\examnum}         & & \\
        \textbf{\examdate}        & & \\
        \textbf{Zeit: \timelimit} & Dozent: & \makebox[2in]{Daniel Senften}
    \end{tabular*}\\
    \rule[1ex]{\textwidth}{1pt}

    Dieser Test besteht aus \numpages\ Seiten (inkl. Titelseite) und
    \numquestions\ Fragen. Wenn alle Fragen richtig beantwortet werden,
    dann sind maximal \numpoints\ möglich.


    \begin{center}
        \textbf{Punktetabelle}\footnote{Diese Tabelle wird durch den Dozenten
        nach der Prüfung ausgefüllt.} \\ \vskip1em
        \addpoints

        \hqword{Aufgabe}
        \hpword{Punkte}
        \htword{\textbf{Total}}
        \hsword{Erreicht}
        \gradetable[h][questions]
    \end{center}

    \noindent
    \rule[1ex]{\textwidth}{1pt}

    % ===========================================================================

    \begin{questions}

        \question[10] Gegeben sind folgende beiden Klassen. Zeichne das vollständige
        Klassen Diagramm (UML).

        \begin{minipage}{0.45\textwidth}
            \inputminted[autogobble,linenos,frame=single]{java}{../java/academy/One.java}
        \end{minipage}
        \begin{minipage}{0.49\textwidth}
            \inputminted[autogobble,frame=single]{java}{../java/academy/Two.java}
        \end{minipage}
        \makeemptybox{5cm}

        \addpoints

        % -----------------------------------------------
        \newpage
        \question Denken Sie an unser Projekt \emph{Figuren zeichnen}, in welchem
        wir ein farbiges Haus (inkl. Sonne) dargestellt und bewegt haben.
        \begin{parts}
            \part[3] Was würde passieren, wenn wir die Methode 'setSize()' mit
            negativen Werten aufrufen?
            \part[3] Ist das eine gute Lösung?
            \part[3] Können Sie sich eine bessere Lösung vorstellen vorstellen?
        \end{parts}


        % setValue()
        \inputminted[autogobble,firstline=31,lastline=39]
        {java}{../java/figures/Triangle.java}

        \ifprintanswers
        \begin{solution}
            \begin{itemize}
                \item Das dargestellte Dach würde auf dem Kopf (auf dem Spitz) stehen.
                \item Nein, gemäss Kommentar dürfen keine negativen Werte übergeben werden.
                \item Es muss ein \emph{ErrorHandling} eingebaut werden.
            \end{itemize}
        \end{solution}
        \else\makeemptybox{2in}
        \fi

        \addpoints

        % -----------------------------------------------
        \question[4] Heute haben wir das Projekt \emph{Zeit} mit der Darstellung einer
        Zeitangabe kennen gelernt. Hierbei kam auch die folgende Funktion vor:

        \inputminted[autogobble,firstline=58,lastline=62]
        {java}{../java/time/NumberDisplay.java}

        Was würde passieren, wenn wir den Oparator \texttt{>=} in der Methode
        \mintinline{java}{setValue()} durch \texttt{>} ersetzen würden, und zwar auf folgende
        Weise:

        \mintinline{java}{if((value > 0) && (value < limit))}

        \ifprintanswers
        \begin{solution}
            Der Wert '0' könnte nie übergeben werden. Dies wird allerdings nicht
            durch eine entsprechende Fehlermeldung unterstützt.
        \end{solution}
        \else\makeemptybox{\fill}
        \fi

        \addpoints

        % -----------------------------------------------
        \newpage
        \checkboxchar{$\Box$} % changing checkbox style locally
        \question[5] Die boolsche Algebra ist ein sehr wichtiger Bestandteil in der
        Softwareentwicklung. Sie kommt in vielen Ausdrücken wie \textbf{for}-Schleifen,
        \textbf{while}-Schleifen$\ldots$ vor.

        Welche der folgenden Ausdrücke liefern den Wert \texttt{true}?

        \addpoints
        \begin{checkboxes}
            \choice \mintinline{java}{! (4 < 5)}
            \CorrectChoice \mintinline{java}{! false}
            \choice \mintinline{java}{(2 > 2) || ((4 == 4) && (1 < 0))}
            \choice \mintinline{java}{(2 > 2) || (4 == 4) && (1 < 0)}
            \CorrectChoice \mintinline{java}{(34 != 33) && ! false}
        \end{checkboxes}

        % -----------------------------------------------
        \question[5] Schreiben Sie einen Ausdruck mit zwei boolschen Variablen \texttt{a}
        und \texttt{b}, der \texttt{true} liefert, wenn nur genau eine von beiden \texttt{true}
        ist, und \texttt{false} liefert, wenn \texttt{a} und \texttt{b} beide \texttt{false}
        oder beide \texttt{true} sind\footnote{Dieser Ausdruck bezeichnet man auch als
        exklusives Oder}.
        \addpoints

        \ifprintanswers
        \begin{solution}
            Die einfachste Lösung ist:
            \mintinline{text}{a != b}
        \end{solution}
        \else\makeemptybox{2in}
        \fi


        % -----------------------------------------------
        \question[10] Beim unserem Projekt \emph{Time} wird die Zeit im Dezimalformat
        dargestellt, wie in \fbox{14:45}. Ändern Sie diese Anzeige in ein binäres
        Format, so dass diese Zeit als \fbox{01110:101101} dargestellt wird.

        Aktuelle Funktion:

        % updateDisplay()
        \inputminted[autogobble,firstline=75,lastline=78]
        {java}{../java/time/Display.java}

        \inputminted[autogobble,firstline=41,lastline=47]
        {java}{../java/time/NumberDisplay.java}


        \addpoints

        \ifprintanswers
        \begin{solution}
            Dies ist die einfachste Variante$\ldots$

            \inputminted[autogobble,firstline=49,lastline=51]
            {java}{../java/time/NumberDisplay.java}
        \end{solution}
        \else\makeemptybox{3in}
        \fi


        % -----------------------------------------------
        \question[10]
        Mit folgender Klasse soll ein Objekt eines bestimmten Datentyps (z.B.
        \mintinline{java}{Integer}) erstellt und anschliessend die Methode
        \mintinline{java}{add()} aufgerufen werden.
        \inputminted[autogobble,linenos,firstline=3]{java}{../java/generics/Cache.java}

        \ifprintanswers
        \begin{solution}
            \begin{minted}[autogobble]{java}
                Cache<Integer> myCache = new Chache<>();
                myCache.add(123); // autoboxing
            \end{minted}
        \end{solution}
        \else\makeemptybox{3in}
        \fi

        % -----------------------------------------------
        \newpage

        \checkboxchar{$\Box$} % changing checkbox style locally
        \question[4] Welches sind korrekte Definitionen der \mintinline{java}{main()} Methode,
        damit das Programm gestartet werden kann.

        \begin{checkboxes}
            \choice \mintinline{java}{public static main(String[] args)}
            \CorrectChoice \mintinline{java}{public static void main(String... args)}
            \choice \mintinline{java}{public void main(String[] args)}
            \CorrectChoice \mintinline{java}{public static void main(String args[])}
        \end{checkboxes}

        \addpoints

        % -----------------------------------------------
        \question[10]
        Objekte und Klassen bauen auf folgenden (nicht abschliessenden) Konzepten auf.
        Was verstehen wir unter den einzelnen Punkten. Gefragt ist eine kurze, stichwortartige
        Umschreibung von:
        \begin{itemize}
            \item Klasse
            \item Objekt
            \item Eigenschaft
            \item Methode
            \item Datentyp
        \end{itemize}
        \makeemptybox{\fill}

        % -----------------------------------------------
        \newpage
        \question[10]
        Gegeben ist die folgende Klasse:
        \inputminted[autogobble,linenos]{java}{../java/academy/calculator/InputValidator.java}

        Was ist das Resultat, wenn ich dieses Programm (Zeile 25) starte:

        \ifprintanswers
        \begin{solution}
            Die Ausgabe ist:

            \mintinline{text}{[-3] Negative Zahlen sind nicht erlaubt.}
        \end{solution}
        \else\makeemptybox{\fill}
        \fi

    \end{questions}

\end{document}
