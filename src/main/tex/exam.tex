\documentclass[12pt,a4paper]{exam}

\RequirePackage{talent-factory}
\RequirePackage[top=3cm,bottom=2cm,left=2cm,right=1.5cm,headsep=10pt,a4paper]{geometry} % Page margins

\newcommand{\class}{Modul 404}
\newcommand{\examnum}{Test 1}
\newcommand{\examdate}{15.02.2019}
\newcommand{\timelimit}{45 Minuten}

\pagestyle{head}
\firstpageheader{}{}{}
\runningheader{\class}{\examnum\ - Seite \thepage\ / \numpages}{\examdate}
\runningheadrule

\begin{document}

    \setmainfont{Verdana}
    \parindent0em\parskip1em

    % Deutsche Schreibweise diverser Labels
    \pointpoints{Punkt}{Punkte}


    \noindent
    \begin{tabular*}{\textwidth}{l @{\extracolsep{\fill}} r @{\extracolsep{6pt}} l}
        \textbf{\class}           & \textbf{Name:} & \makebox[2in]{\hrulefill} \\
        \textbf{\examnum}         &                & \\
        \textbf{\examdate}        &                & \\
        \textbf{Zeit: \timelimit} & Dozent         & \makebox[2in]{Daniel Senften}
    \end{tabular*}\\
    \rule[1ex]{\textwidth}{1pt}

    Dieser Test besteht aus \numpages\ Seiten (inkl. Titelseite) und
    \numquestions\ Fragen. Wenn alle Fragen richtig beantwortet werden,
    dann sind maximal \numpoints\ möglich..


    \begin{center}
        \textbf{Punktetabelle}\footnote{Diese Tabelle wird durch den Dozenten
        nach der Prüfung ausgefüllt} \\ \vskip1em
        \addpoints

        \hqword{Aufgabe}
        \hpword{Punktzah}
        \htword{\textbf{Total}}
        \hsword{Erreicht}
        \gradetable[h][questions]
    \end{center}

    \noindent
    \rule[1ex]{\textwidth}{1pt}

    \begin{questions}

        \question[1] Calculate 2+2.
        \addpoints

        \question[20] Consider the function $f(x)=3x^3+2x^2+x+1$.
        \noaddpoints % to omit double points count
        \begin{parts}
            \part[10] Calculate $f'(x)$.
            \part[10] Calculate $f''(x)$.
        \end{parts}
        \addpoints

        \question[2] One of these things is not like the others; one of these
        things is not the same. Which one is different?
        \begin{choices}
            \choice John
            \choice Paul
            \choice George
            \choice Ringo
            \choice Socrates
        \end{choices}

        \question[2] One of these things is not like the others; one of these
        things is not the same. Which one is different?
        \begin{oneparchoices}
            \choice John
            \choice Paul
            \choice George
            \choice Ringo
            \choice Socrates
        \end{oneparchoices}

        \question[3] Mark box if true.
        \addpoints
        \begin{checkboxes}
            \choice 2+2=4
            \choice $\frac{d}{dx} (x^2+1) = 2x+1$
            \choice The Moon is made of cheese.
        \end{checkboxes}

        {%
        \checkboxchar{$\Box$} % changing checkbox style locally
        \question[4] Welches sind korrekte Definitionen der \mintinline{java}{main()} Methode,
        damit das Programm gestartet werden kann
        \addpoints
        \begin{checkboxes}
            \choice \mintinline{java}{public static main(String[] args)}
            \choice \mintinline{java}{public static void main(String... args)}
            \choice \mintinline{java}{public void main(String[] args)}
            \choice \mintinline{java}{public static void main(String args[])}
        \end{checkboxes}
        }%

        {%
        % changing choice items style locally
        \renewcommand*\thechoice{\arabic{choice}}
        \renewcommand*\choicelabel{\thechoice)}
        %
        \question[2] Element with $Z=92$ is:
        \begin{multicols}{2}
            \begin{choices}
                \choice H
                \choice O
                \choice F
                \choice S
                \choice Ba
                \choice Pb
                \choice U
                \choice Pu
            \end{choices}
        \end{multicols}
        }%

        \question[10]
        Mit folgender Klasse soll ein Objekt eines bestimmten Datentyps (z.B.
        \mintinline{java}{Integer}) erstellt und anschliessend die Methode
        \mintinline{java}{add()} aufgerufen werden.
        \inputminted[autogobble,linenos,firstline=3]{java}{../java/generics/Cache.java}
        \makeemptybox{2in}

        \newpage
        \question[10]
        Objekte und Klassen bauen auf folgenden (nicht abschliessenden) Konzepten auf.
        Was verstehen wir unter den einzelnen Punkten. Gefragt ist eine kurze, stichwortartige
        Umschreibung von:
        \begin{itemize}
            \item Klasse
            \item Objekt
            \item Eigenschaft
            \item Methode
            \item Datentyp
        \end{itemize}
        \makeemptybox{\fill}

        \newpage\question[10]
        Gegeben ist die folgende Klasse:
        \inputminted[autogobble,linenos]{java}{../java/academy/calculator/InputValidator.java}

        Was ist das Resultat, wenn ich dieses Programm (Zeile 25) starte:
        \makeemptybox{\fill}

    \end{questions}

\end{document}
